\documentclass{beamer}


\usepackage[ngerman]{babel}
\usepackage[latin1]{inputenc}

\usetheme{Warsaw}
\usecolortheme{beaver}
 
\title{Git}
\author{Sascha Graeff, Felix Wiedemann}
\date{\today}
 
\begin{document}
\maketitle
\frame{\tableofcontents}

\section{Versionskontrollsysteme}
\begin{frame}
  \frametitle{Versionskontrollsysteme}
  \begin{itemize}
     \item Definition: Versionskontrollsystem (VCS) \\
           Ein Werkzeug, dass das Kontrollieren unterschiedlicher Versionen eines Projektes erm�glicht und vereinfacht.
  \end{itemize}
\end{frame} 

\section{Vorteile von Versionierung}
\begin{frame}
  \frametitle{Warum Versionierung nutzen?}
  \begin{itemize}
    \item Im Team zusammen an einem Projekt arbeiten
    \item Protokollierung der Entwicklung (Code-Historie)
    \item Vereinfachtes Zusammenf�hren mehrer Code-Versionen
    \item Mehrere Entwicklungszweige verwalten
  \end{itemize}
\end{frame}

\section{Versionskontrollsysteme}
\begin{frame}
  \frametitle{Die zwei beliebtesten Versionskontrollsysteme}
  \begin{itemize}
     \item Subversion
     \item Git
  \end{itemize}
\end{frame} 

\section{Git: Struktur}
\begin{frame}
  \frametitle{Struktur eines Git-Netzwerks}
\end{frame}

\section{Workflow}
\begin{frame}
  \frametitle{Workflow} \pause
  \begin{itemize}
    \item Updates pullen \pause
    \item Auf neuen feature-Branch wechseln, der vom master abzweigt \pause
	\item Dinge tun \pause
	\item Committen \pause
	\item Auf master-Branch zur�ck wechseln \pause
	\item feature-Branch in den master-Branch mergen \pause
	\item (Pullen) \pause
	\item �nderungen pushen
  \end{itemize}
\end{frame}

\section{eGit}
\begin{frame}
  \frametitle{eGit}
\end{frame}

\end{document}

